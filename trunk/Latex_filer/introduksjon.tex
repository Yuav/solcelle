\section{Introduksjon}

Solceller er antatt � dominere energisektoren de neste hundre �r. For at dette skal bli tilfelle trengs det billige og effektive solceller. Multikrystallinsk silisium er materialet som har mest potensiale for � oppn� dette. Det er billig � produsere, men har ogs� relativt lav utnyttelse av solenergien. Derfor er det viktig � n�yaktig kunne identifisere kilder til tap, og forst� virkem�ten til slike materialer. 

Det er oppdrettet et laboratorium for � kunne gj�re m�linger p� slike celler ved hjelp av fotoluminisens p� ekstremt lave temperaturer. Tidligere m�linger av multikrystallinsk silisium p� dette laboratoriet viser deler av et spekter som er � finne p� tilsvarende m�linger (f.eks \ref{tarasov00}), men deler av tapsspekteret som var forventet dukket ikke opp. Dette prosjektet fokuserer p� hva som er �rsaken til dette avviket, og hvordan det kan utbedres. I tillegg til det er det fokusert p� virkem�te til solceller, og kilder til tap i multikrystallinsk silisium.
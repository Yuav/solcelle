\subsection{Solceller}

En halvleder med et p-dopet og et n-dopen omr�de som ligger inntil hvernadre kalles en pn-overgang. En slik pn-overgang har likerettende egenskaper. Det vil si at den leder str�m vesentlig bedre i den ene retningen enn den andre. Denne oppf�rselen definerer en diode. Siden p-siden har en konsentrasjon av elektroner i ledningsb�ndet som er vesengli lavere enn n-sidens konsentrasjon av elektroner i ledningsb�ndet, vil vi f� en transport av ledningsb�nd-elektroner fra n.siden til p-siden ved diffusjon. Det samme skjer ogs� for hull fra p-siden til n-siden. Denne str�mmen av ladnings kalles diffusjonsstr�mmen. I prinsippet kan ogs� dopantene Si, B og P diffundere mellom de to delene av krystallen, men er bare betydelig for veldig h�ye tempraturer, alts� ikke vesengig i romtempratur.

% figur fra s. 14 i kompendiet

Deplesjonssjiktet er omr�det n�r grenseflaten mellom de to dopekonsentrasjonene som vil v�re essensielt t�mt for frie ladningsb�rere. Siden n-siden av deplesjonssjiktet inneholder donorer uten tilh�rende elektron vil denne siden v�re positivt ladet, og tilsvarende vil p-siden v�re negativt ladet. Dette gj�r at det oppst�r et elektrisk felt fra n- til p-siden, eller et fall i potensial fra n-siden til p-siden. Dette feltet f�rer til en driftstr�m som g�r i motsatt retning av diffusjonsstr�mmen og f�rer til 0 netto str�m, eller likevekt.

Ved belysning genereres det minoritetsb�rere i pn-overgangen utover de som genereres termisk ved at fotoner eksiterer elektroner til ledningsb�ndet. Denne generingen er ofte vesentlig st�rre enn driftstr�mmen. Denne str�mmen er uavhengig av potensialforskjellene i pn-overgangen. For en diode i m�rke er det gitt en str�m-spenning karakteristikk gitt av:

\begin{equation}
I=|I_drift|e^{\frac{qV}{kT}-1}
\label{eq:diodeiv}
\end{equation}

N�r pn-overgangen blir belyst vil driftstr�mmen �ke, og forskyve str�m-spenning karakteristikken nedover

% figur fra side 22 i kompendiet

For solceller defineres ofte str�m ut av cellen som positiv, slik at karakteristrikken vendes om V-aksen

\begin{equation}
I=I_belysning-I_drift(e^{\frac{qV}{kT}-1)}
\label{eq:solcelleiv}
\end{equation}

hvor $I_belysning$ er str�m generert av lys. Spenningen ved �pen krets er gitt ved:

\begin{equation}
V_OC=\frac{kT}{q}lan(\frac{I_belysning}{I_drift}+1)
\label{eq:voc}
\end{equation}

Maks effekt som genereres av solcellen er gitt av:

\begin{equation}
P_m=I_m V_m
\label{eq:piv}
\end{equation}

Hvor $P_m$ er maks effekt, $I_m$ er maks str�m og $V_m$ er maks spenning. Fyllfaktoren FF er gitt av faktisk effekt ut, over teoretisk maks effekt:

\begin{equation}
FF=\frac{I_m V_m}{I_belysning V_OC}
\label{eq:fyllfaktor}
\end{equation}

% fyllfaktor figur

Virkningsgraden til en solceller er representert ved \nu, som er gitt ved:

\begin{equation}
\nu=\frac{P_m}{P_inn}=FF \frac{I_belysning V_OC}{P_inn}
\label{eq:virkningsgrad}
\end{equation}
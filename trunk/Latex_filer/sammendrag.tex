%\section*{Sammendrag}

Tidligere fotoluminiscens m�linger gjort av Sintef p� multikrystallinsk silisium mangler s�kalte D-Linjer, som relaterer seg til dislokasjonslinjer og defekter i pr�ven. Disse defektene er kilder til tap for solceller. Det viser seg at det er flere kilder til tap i laboppsettet som ble tatt i bruk for b�lgelengder rundt 1550nm, eller 0.8eV. Det er disse b�lgelengdene som er mest interesante med tanke p� � kunne karakterisere tap i en pr�ve. Ved � utbedre laboppsettet fikk man mer enn tre ganger s� mye lys p� b�lgelengde 1530nm fram til spektrometeret.

Ved � se p� ulike posisjoner p� en pr�ve av multikrystallinsk silisium er det p�vist en posisjonsavhengighet p� spekteret. Ytterpunktene p� disse posisjonene er i et s�kalt bra omr�de, og et d�rlig omr�de som har kilder til tap, som defekter.

Nye m�linger gjort p� pr�ven viser et spekter med dislokasjonslinjer, som ikke har v�rt observert ved hjelp av det gamle oppsettet. Linjene som er omtalt som D3 og D4 kommer til syne. Linjene som omtales som D1 og D2, blir ikke observert. 
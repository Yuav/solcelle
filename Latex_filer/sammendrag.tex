%\section*{Sammendrag}

Det er nyttig � kunne karakterisere ulike former for tap i solcellematerialer, slik at det er mulig � forbedre dem. En m�te � gj�re dette p� er m�linger basert p� fotoluminiscens. Tidligere fotoluminiscens m�linger gjort av Sintef p� multikrystallinsk silisium mangler s�kalte D-Linjer, som relaterer seg til dislokasjonslinjer og defekter for multikrystallinsk silisum. Disse defektene er kilder til tap for solceller. �rsaken til at ikke disse linjene er observert antas � ha med tap i selve labutstyret � gj�re.

Det viser seg at det er flere kilder til tap i laboppsettet som ble tatt i bruk for b�lgelengder rundt 1550nm, eller 0,8eV. Det er disse b�lgelengdene som er mest interesante med tanke p� � kunne karakterisere tap i en pr�ve med silisium \cite{tarasov00}. Ved � utbedre laboppsettet kom det mer enn tre ganger s� mye lys p� b�lgelengde 1530nm fram til spektrometeret.

Ved � se p� ulike posisjoner p� en pr�ve av multikrystallinsk silisium er det observert en posisjonsavhengighet p� spekteret. Ytterpunktene p� disse posisjonene er i et s�kalt bra omr�de, og et d�rlig omr�de. Et d�rlig omr�de har kilder til tap, som defekter, mens et bra omr�de hovedsaklig best�r av intrinsikk silisium.

Nye m�linger gjort p� en pr�ve med multikrystallinsk silisium ved lavtemperatur p� et d�rlig omr�de viser et spekter med dislokasjonslinjer, som ikke har v�rt observert ved hjelp av det gamle oppsettet. Linjene som er omtalt som D3 og D4 kommer til syne, samt antydning til en ukjent karakteristikk ved 0,98eV. Linjene som omtales som D1 og D2, blir ikke observert. �rsaken til manglende D1 og D2 linjer er ikke kjent. For et bra omr�de stemmer spekteret mye bedre med forventede verdier \cite{davies88}.
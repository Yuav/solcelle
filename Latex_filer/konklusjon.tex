\section{Konklusjon}

Oppsettet fra tidligere har store tap for b�lgelengder rundt 1550nm, eller 0,8eV som er relatert til dislokasjoner og defekter for multikrystallinsk silisium. Ved � sette opp en parallell veibane til bruk for disse b�lgelengdene kommer det fram over tre ganger s� mye lys til spektrometeret, enn gjennom det gamle oppsettet ved 1540nm. Det er p�vist en posisjonsavhengighet p� spekteret. Ytterpunktene befinner seg i et s�kalt bra omr�de, og et d�rlig omr�de med defekter og dislokasjoner.

Nye m�linger p� et bra omr�de viser en karakteristikk som stemmer bra med tidligere publikasjoner, og viser kjente fotoluminescens spekter. 

Nye m�linger for et d�rlig omr�de f�r fram et spekter som er relatert til dislokasjonslinjer. Det er kun D3 og D4 linjer som er synlige, D1 og D2 er ikke observert. �rsaken til frav�ret av D1 og D2 er ikke kjent, og b�r unders�kes n�rmere. Det er ogs� kommet fram en topp som ofte er relatert til forurensinger av boratomer. Spekteret med dislokasjonslinjer har ikke v�rt synlig tidligere ved denne labben. Proskjektet er i s� m�te et steg framover.